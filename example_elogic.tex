\documentclass[11pt]{amsart}
\usepackage{geometry}                % See geometry.pdf to learn the layout options. There are lots.
\geometry{letterpaper}                   % ... or a4paper or a5paper or ... 
%\geometry{landscape}                % Activate for for rotated page geometry
%\usepackage[parfill]{parskip}    % Activate to begin paragraphs with an empty line rather than an indent
\usepackage{graphicx}
\usepackage{amssymb}
\usepackage{epstopdf}
\usepackage{algorithmic}
\usepackage{ifthen}
\usepackage{elogic}
\DeclareGraphicsRule{.tif}{png}{.png}{`convert #1 `dirname #1`/`basename #1 .tif`.png}

\title{Demonstration of E-Logic proofs}
\author{Simon Hudon}
\date{}                                           % Activate to display a given date or no date

\begin{document}
\maketitle

\section{Introdution}
This is an example to demonstrate the formatting of proofs in e-logic.  For that purpose, I prove the partial correctness of a program computing the $gcd$ function.

\section{Useful Properties}

These are the free variables used in the discussion:
\begin{align*}
	x, y, m, r, q: &\mathbb{N} \\
\end{align*}

And here is the definition of $\gcd$ that we use:

\begin{align}
	& \gcd(x,y) | x \\
	& \gcd(x,y)| y \\
	& \iter{\forall}{m}{m|x\,\land\,m|y}{m \le \gcd(x,y)}
\end{align}

Finally, here is a couple of predicates that we use to postulate the definition of the division and modulo.

\begin{align}
	& (x = q \cdot y + r) \; \land \; (r < y) \\ 
	& (q = x \div y) \; \land \; (r = x \mod y)
\end{align}

We use them to define $\div$ and $\mod$: $(4) \equiv (5)$.

\section{Program 1}

For lack of time tonight, we simply show an algorithm to compute both the quotient and the remainder of x and y.  We use formula (4) as a termination condition.  More precisely, the left conjunct is the invariant and the right conjunct is the exit condition.  At the

\begin{algorithmic}
	\STATE $r := x$
	\STATE $q := 0$
	\REPEAT
		\STATE $r, q := r - y, q + 1$
		\STATE \COMMENT {$x = q \cdot y + r$}
	\UNTIL{$r < y$}
\end{algorithmic}

Therefore, we want to prove:

\begin{equation}
	 (x = q \cdot y + r) \; \land \; (r < y) \implies  x = (q + 1) \cdot y + (r - y)
\end{equation}

\begin{proof} 

	\begin{deriv}
		\step{\"(q + 1) \cdot y\" + (r - y)}
			\infer{=}{Distributivity}
		\step{\*q \cdot y + y \* +  \" (r - y) \" }
			\infer{=}{$a-b=a+(-b)$; commutativity}
		\step{q \cdot y + \" y + \*- y \" + r \* }
			\infer{=}{ $a+(-a) = 0$}
		\step{q \cdot y + \* r \*}
			\infer{=}{Hypothesis}
		\step{x}
	\end{deriv}
\end{proof}

\section{Program 2}

This is the gcd program.  We will use

\begin{equation}
	\gcd (x, y) = \gcd (w, z)
\end{equation}

as our invariant.  The termination condition will be $w = z$.  At that point, we use the following property

\begin{equation}
	\gcd (w, w) = w 
\end{equation}

To prove that property, we can use the following definition of the 'divides' operator:

\begin{equation}
	a|b \ \equiv \ \iter{\exists}{c}{}{c \cdot a = b}
\end{equation}

\begin{proof} of (8)
	\begin{deriv}
		\step{\gcd(w,w) = w}
			\infer{\equiv}{Definition of $\gcd$} 
		\step{  \"    w|w \, \land \, w|w  \" \, \land \, 
				\iter{\forall}{m}{m|w \, \land \, m|w}{m \le \gcd (w,w)}}
			\infer{\equiv}{Idempotency of $\land$}
		\step{ \*    w|w  \*  \, \land \, 
				\"    \iter{\forall}{m}{m|w \, \land \, m|w}{m \le \gcd (w,w)}   \"  }
			\infer{\equiv}{Trading}
		\step{ \"  w|w  \" \, \land \, 
				\*    \iter{\forall}{m}{}{ \"   m|w \, \land \, m|w \implies m \le \gcd (w,w)}    \" \* }
			\infer{\equiv}{Leibniz twice}
		\sidederiv{
			\step{a}
				\infer{\equiv}{Definition of $|$}
			\step{
				\iter{\exists}{c}{}{c \cdot a = b}
			}
		}
	\end{deriv}
\end{proof}


\begin{algorithmic}
	\STATE $w, z := x, y$
	
\end{algorithmic}

\end{document}  